\documentclass[a4paper,12pt]{article}
\usepackage[italian]{babel}
\usepackage[utf8]{inputenc}

% Larger borders -- we do not want do waste paper, even if it is only paper on screen =)
\usepackage[top=2.5cm, bottom=2.5cm, left=2cm, right=2cm]{geometry}
% Remove auto indentation of paragraphs.
\setlength\parindent{0pt}

% Palatino font (nicer serif font: Times is for oldies)
\renewcommand*\rmdefault{ppl}

% Nested itemize list bullet style
\renewcommand{\labelitemi}{$\bullet$}
\renewcommand{\labelitemii}{$\circ$}
\renewcommand{\labelitemiii}{--}

% Math packages
\usepackage{amsmath}
\usepackage{amsfonts}
\usepackage{amssymb}

% Graphic packages
\usepackage{graphicx}
\usepackage{float}
\usepackage{adjustbox}
\usepackage{tikz}
\usepackage{forest,array}
\usetikzlibrary{shadows}

% Graphs styles
\forestset{
  giombatree/.style={
    for tree={
      grow = east,
      parent anchor=east,
      child anchor=west,
      edge={rounded corners=2mm},
      fill=violet!5,
      drop shadow,
      l sep=10mm,
      edge path={
        \noexpand\path [draw, \forestoption{edge}] (!u.parent anchor) -- +(5mm,0) -- (.child anchor)\forestoption{edge label};
      }
    }
  }
}
\forestset{
  qtree/.style={
    for tree={
      parent anchor=south,
      child anchor=north,
      align=center,
      edge={rounded corners=2mm},
      fill=violet!5,
      drop shadow,
      l sep=10mm,
    }
  }
}


% Hides ugly links from the index
\usepackage[hidelinks]{hyperref}
% Landscape format pdf pagess
\usepackage{pdflscape}

\title{Cybersecurity Project}
\author{F. Barbarulo, G. B. Rolandi}

\begin{document}
\maketitle
\tableofcontents

\clearpage

\section{Protocol}
Client must create a secure communication channel to connect to the server.
% TODO more details on next lessons

A \texttt{parola} is defined as a pure ASCII sequence which can match the following regular expression \texttt{[A-Za-z0-9\textbackslash.\_]*}.
Different \texttt{parola}s can be separed by a whitespace.

\subsection{Request commands}
A request command is sent by the client to the server.

Generic structure of a request command is:
\begin{verbatim}
COMM filename\n
tag: value\n
\n
body
\end{verbatim}

Each line is terminated by a \texttt{LF} char, and each command is terminated by a \texttt{LF} char.
It is important to send the line feed char because server will not parse the command until two trailing \texttt{LF}s are received.

Note: a \texttt{LF} char is represented by a byte whose value is \texttt{0x0A}.
\\
A command is composed by the following parts:
\begin{itemize}
  \item \texttt{COMM} is a 4 char \texttt{parola} and represents the command to issue;
  \item \texttt{filename} is the name of a file on which issue the command.
  Filenames are composed by one \texttt{parola}.
  Depending on the specific command, a filename is mandatory or forbidden;
  \item \texttt{tag: value} is a pair of two \texttt{parola}s, composed by a \texttt{parola} immediately followed by a colon, a whitespace and another \texttt{parola}.
  First \texttt{parola} represents the tag for an additional parameter of current command, while second one representes the actual value for the parameter; depending on the specific command, one or more parameters may be mandatory or forbidden;
  \item \texttt{body} is a stream of bytes whose length must be specified in a command parameter; depending on the specific command, a body is mandatory or forbidden:
\end{itemize}

\subsection{Reply status}
A reply status is sent by the server to the client.

Generic structure of a reply status is:

\begin{verbatim}
DDD\n
tag: value
\n
body
\end{verbatim}

A reply status is composed by the following parts:
\begin{itemize}
  \item \texttt{DDD} a 3 ASCII digits number representing the state of the previous request command;
  \item \texttt{tag: value} is a pair to represent a parameter, with the same format of the request command parameter; depending on the specific command, one or more parameters may be mandatory or forbidden;
  \item \texttt{body} is a stream of bytes whose length must be specified in a reply status parameter; depending on the specific command the reply refers to, a body is mandatory or forbidden;
\end{itemize}

\subsection{Request commands list}
\subsubsection{ALLO}
\texttt{ALLO filename}

Request server to allocate a file called \emph{filename}.
This command is useful to atomically:
\begin{itemize}
  \item test if a file exists on the remote server
  \item create that file
\end{itemize}
and must be used before issuing a \texttt{STOR} command.

Reply status:
\begin{itemize}
  \item \texttt{200}: ok
  \item \texttt{201}: ok, file already exists
  \item \texttt{452}: bad file
\end{itemize}

\subsubsection{DELE}
\texttt{DELE filename}

Reply status:
\begin{itemize}
  \item \texttt{200}: ok
  \item \texttt{452}: bad file
\end{itemize}

\subsubsection{LIST}
\texttt{LIST}

Request the server to send a list of available files

Reply status:
\begin{verbatim}
200\n
Size: listsize\n
\n
body
\end{verbatim}

\subsubsection{QUIT}
\texttt{QUIT}

Close connection with the server.

Reply status:
\begin{itemize}
  \item \texttt{200}: ok
\end{itemize}

\subsubsection{RETR}
\texttt{RETR filename}

Retrieve a file from the server.

Reply status:
\begin{itemize}
  \item success:
\begin{verbatim}
200\n
Size: filesize\n
\n
body
\end{verbatim}
  \item failure:\\
  \texttt{452}: bad file
\end{itemize}

\subsubsection{STOR}
\begin{verbatim}
STOR filename
Size: filesize

body
\end{verbatim}

Store a file on the remote server.
This command can be issued only after an \texttt{ALLO} command.

\begin{itemize}
  \item \texttt{Size: filesize} is a parameter pair. \texttt{filesize} is the decimal ASCII representation of the file size in bytes;
  \item \texttt{body} is the actual content of the file;
\end{itemize}

Reply status:
\begin{itemize}
  \item \texttt{200}: ok
\end{itemize}

\subsection{Generic return values}
\begin{itemize}
  \item \texttt{500}: syntax error
\end{itemize}

\end{document}
